\chapter{Introduction}
In an era where precise positioning, navigation, and timing have become fundamental requirements for numerous applications, the \textit{Global Navigation Satellite System} (GNSS) has emerged as a transformative technology. \textit{Global Positioning System}, commonly known as GPS, has revolutionized the way we navigate, communicate, and interact with our surroundings. As the demand for accurate and reliable GNSS positioning continues to grow, there is an increasing need to explore innovative methods to develop advanced GNSS receivers capable of handling diverse and challenging environments.

The aim of this project is to design and develop a GPS/GNSS receiver, focusing on the critical component of correlators, which form the backbone of any GNSS receiver. Correlators play a vital role in the tracking part of the GNSS positioning process, making them a key element in the overall receiver architecture.

\section{Background}
The GNSS has become an integral part of modern life, transforming how we navigate, communicate, and interact with the world around us. As the demand for accurate and reliable positioning information continues to grow, there is a constant need for advancements in GNSS receiver technology. The receiver's tracking module, responsible for extracting precise positioning data from satellite signals, plays a critical role in ensuring the accuracy and robustness of the overall GNSS positioning process.

\begin{itemize}
    \item Increasing Reliance on GNSS: In recent years, there has been a substantial increase in the reliance on GNSS technology across various sectors and industries. From commercial transportation and aviation to precision agriculture, emergency response, and scientific research, GNSS has become a ubiquitous tool in countless applications. The accurate positioning and timing information provided by GNSS have not only improved operational efficiency but have also enabled the development of innovative services and solutions.
    \item Importance of the Tracking Module: The tracking module is a vital component of any GNSS receiver, responsible for acquiring and locking onto the weak satellite signals and maintaining their synchronization. Efficient tracking algorithms can increase SNR (\textit{Signal-to-Noise-Ratio}) which are essential for accurate and real-time positioning, especially in dynamic environments where satellite signals may be temporarily obscured or weakened. Optimizing the tracking module's performance is crucial to enhancing the overall receiver's sensitivity, accuracy, and responsiveness to varying environmental conditions.
    \item Opportunities for Hardware Optimization: As the demand for high-performance GNSS receivers increases, there are opportunities to explore hardware optimization techniques to improve tracking capabilities. Implementing the tracking module using VHDL (\textit{Very High-Speed Integrated Circuit Hardware Description Language}) offers advantages such as parallel processing and hardware acceleration, allowing for real-time tracking and reduced power consumption. The utilization of VHDL enables the development of customized \textit{Application-Specific Integrated Circuits} (ASICs) or \textit{Field-Programmable Gate Arrays} (FPGAs) tailored to the tracking function, leading to more efficient and specialized GNSS receivers.
\end{itemize}

\section{Aims and Objectives}
\subsection{Aims}
The project has three aims: first, to design a tracking processor for GNSS receivers using VHDL; second, to verify the results using MATLAB; and third, to learn about GNSS signal acquisition and tracking by reading papers, books, etc.

\subsection{Objectives}
\begin{itemize}
    \item Literature survey on GNSS signal processing and tracking
    \item Be familiar with the principles of correlation functions, correlators, VHDL and Vivado
    \item Write a literature review
    \item Create and verify a set of correlators model in MATLAB
    \item Write the RTL (\textit{Register Transfer Level}) code of the correlator in VHDL
    \item Write the testbench code of the correlator in VHDL
    \item Instance several times (e.g. three times) of the correlator VHDL code to implement the tracking part of the GNSS receiver
    \item Write the testbench code of the project
    \item Compare the results from VHDL with those from MATLAB
    \item Record and analyse the VHDL results against the MATLAB results
    \item Complete the final dissertation
\end{itemize}

\section{Description of the Work}
I will design the correlators of the receiver using VHDL.I plan to design a set of correlators in an FPGA. The correlators can be used to capture GNSS signals to allow subsequent devices to perform ranging and positioning. My front-end device will acquire the GNSS signal and transmit it to me. During the acquisition process I can receive at least one GNSS signal. I will extract the ranging code from it and compare it with the ranging code generated by the receiver, i.e. using a correlator.




