\cleardoublepage
\phantomsection

\addcontentsline{toc}{chapter}{Abstract}

\begin{abstract}

This thesis presents a comprehensive exploration of the design and implementation of a tracking module for GPS signal reception using FPGA technology. The project encompasses a theoretical foundation, practical development, and insightful reflections on challenges and future directions.

The work begins with an overview of the project's objectives and the rationale for employing FPGA technology in GPS signal tracking. It delves into the intricacies of GPS signal characteristics, reviewing relevant literature, and establishing a theoretical foundation for the project.

The core of the thesis focuses on the practical implementation of the tracking module. It includes the selection of hardware components, development environments, and detailed signal analysis. Key modules, such as the NCO, code generator, and correlator, are thoroughly examined and their designs explained.

Results obtained from simulations and waveform analyses are presented and compared to theoretical expectations. Challenges encountered during the development process are highlighted, particularly in debugging and the alignment of theory with practice.

Looking forward, potential areas for improvement are outlined, including enhancing NCO accuracy, refining the correlator's performance, and developing more objective evaluation methods. The use of version control software (Git) is also noted as a valuable practice for managing complex projects.

In summary, this thesis offers a holistic perspective on GPS signal tracking, from theoretical foundations to practical implementation and reflections on challenges and future work. It contributes to the field by providing insights into the complexities of GPS signal tracking and sets the stage for further advancements in this domain.

\end{abstract}