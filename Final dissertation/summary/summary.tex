\myClearDoublePage
\chapter{Summary and Reflections}
As the results came in, the project came to a close. There is a lot of work to summarize and reflect on as well as improve. I will describe two aspects below.

\section{Summary and Reflections}
For the thesis: the first chapter focuses on the background as well as the objectives of the project. The second chapter introduces the related work, starting with the signal and analysing the GPS/GNSS signal characteristics, followed by reviewing the previous books related to GPS tracking, and finally introduces the FPGA technology and explains the reason for using FPGA. Chapter 3 analyses theoretically how to perform GPS signal tracking, various modules such as NCO, code generator etc. are studied and analysed. Finally, the verification methodology is explained to ensure that the results can be evaluated objectively. Chapter 4 is a concrete implementation, where much of the theory needs to be engineered due to the gap between theory and practice. Firstly, the equipment used, and the development environment are described. Next the input signals are analysed. Finally, each module was developed, and the core algorithms in it were introduced. In the last chapter, the results are analysed, and the results are predicted from the theory first, and then compared with the actual results, which are basically in line with the expectations.

For the project itself, I think it is somewhat difficult. Especially in terms of code debugging. I need to analyse the errors from the waveforms and this is something I need to improve. I also realized the difference between theory and practice, and that completing the code writing does not necessarily complete the function in question.

\section{Future Work}
In section \ref{sec:results_waveform}, I mentioned that the correlator's performance is not up to standard. I think the following aspects can be improved:

\begin{itemize}
    \item NCO accuracy needs to be improved. When reviewing the simulation waveforms, I found that the frequency generated by the NCO is not accurate, I suspect that it is a precision issue, and subsequently the NCO module can be redesigned to improve the precision.
    \item For the problem that the peak value is not obvious enough. I suspect that the correlator design is unreasonable. Subsequently, I plan to export the middle data to \textit{SoftGNSS} for testing, and compare each step of the correlator to determine where the problem lies.
    \item More objective evaluation methods are needed. However, there is currently no such method, and the commonly used C/No plot, which is the ratio of carrier signal energy to noise energy over a 1 Hz bandwidth, which is related to multipath effects, receiver antenna gain, attenuation of the antenna cable, the level of the satellite signal emission, and tropospheric delays, can be used, and is usually set to a value of 45 dB-Hz \cite{RN211}.
\end{itemize}
