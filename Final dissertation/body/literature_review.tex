\chapter{Literature Review}

The literature review chapter serves as a critical foundation for this project, offering an in-depth exploration of the existing research, developments, and advancements in the field of GNSS receivers, with a particular focus on the tracking module. This chapter aims to identify the key theories, methodologies, and technologies that have shaped the evolution of GNSS tracking and provide valuable insights for the design and development of the FPGA-based GNSS tracking module.

\section{Interface Control Document\texorpdfstring{\cite{RN170, RN171, RN172, RN174, RN173, RN175}}{}}
In complex engineering projects and systems development, clear and effective communication between various subsystems and components is essential for successful integration and operation. The \textit{Interface Control Document} (ICD) of GNSS provides the basic information of the system, such as intermediate frequency, modulation scheme, code frequency , etc. 

After reviewing the interface control documents of some systems, I have summarised the parameters of each system in Tbl. \ref{tab:spec_gnss}. 

\begin{landscape}
% \vspace*{\fill}
\begin{center}
\begin{table}
    \centering
    \caption{Specifications of Systems}\label{tab:spec_gnss}
    \begin{tabular}{ccccc}
    \toprule
    Specifications & GPS & Glonass & Galileo & BeiDou\\
    \midrule
    \thead{Frequenc\\band} & \thead{L1: 1575.42 MHz\\L2: 1227.6 MHz\\L5: 1176.45 MHz} & \thead{L1: 1602MHz\\L2: 1246MHz\\(14 channels)} & \thead{E1: 1575.420MHz\\E6: 1278.750MHz\\E5a: 1176.450MHz\\E5b: 1207.140MHz} & \thead{B1c : 1575.42MHz\\B2a: 1176.45MHz\\B2b: 1207.14MHz\\B1I: 1561.098MHz\\B3I: 1268.52MHz}\\
    \thead{Band\\width} & \thead{Block IIR, IIRM, and IIF:\\20.46 MHz\\GPS III, GPS IIIF, and subsequent:\\30.69 MHz} & \thead{L1: 7.875MHz(562.5 kHz each)\\L2: 6.125MHz(437.5 kHz each)} & \thead{E1: 24.552MHz\\E6: 40.920MHz\\E5a: 20.460MHz\\E5b: 20.460MHz} & \thead{B1c: 32.736MHz\\B2a: 20.46MHz\\B2b: 20.46MHz\\B1I: 4.092MHz\\B3I: 20.46MHz}\\
    \thead{Modulation\\scheme} & BPSK & \thead{Modulo-2 addition\\CDMA}& \thead{E1: CBOC\\E5: AltBOC\\E6: BOC} & \thead{B1c: QMBOC(6, 1, 4/33)\\Others: BPSK}\\
    \thead{Antenna\\polarization} & \multicolumn{4}{c}{RHCP*}\\
    \thead{Chip\\rate} & \thead{L1 C/A \& P: 1.023MHz\\L2 CL \& CM: 511.5 kHz\\L5 data \& channel: 10.23 MHz} & \thead{L1 C/A: 0.511MHz\\L1 P: 5.11MHz\\L2 C/A: 0.511MHz\\L2 P: 5.11MHz} & \thead{E1 ranging Code: 1.023MHz\\E6: 5.115MHz\\E5: 10.230MHz} & \thead{B1c ranging code: 1.023MHz\\B2a ranging code: 10.23MHz\\B2b ranging code: 10.23MHz\\B1l ranging code: 2.046MHz\\B3l ranging code: 10.23MHz}\\
    \bottomrule
    \end{tabular}
    \footnotesize Note: *RHCP: \textit{Right Hand Circularly Polarized}
\end{table}
\end{center}
% \vspace*{\fill}
\end{landscape}
Given that the GPS system was the first to be deployed, it possesses a relatively straightforward structure. As a result, the GPS L1 signal has been selected as the primary target signal for this project. Furthermore, it is important to highlight that the L5 signal exhibits ten times the bandwidth of L1. Thus, in the event that the utilization of the L5 signal becomes necessary, the adjustments required for its implementation would be minimal.

\section{Book from P. D. Groves\texorpdfstring{\cite{RN178}}{}}
This book presents a comprehensive textbook that covers navigation systems, encompassing GNSS, \textit{Inertial Navigation Systems} (INS), and other sensors to deliver precise \textit{positioning, navigation, and timing} (PNT) information. The esteemed authors behind this work are Paul D. Groves and Kayton M. Ned, both recognized experts in the field of navigation systems.

Paul D. Groves holds the position of a professor of satellite navigation at the University College London, where his research focuses on advanced navigation algorithms and technologies. On the other hand, Kayton M. Ned, a former professor of aeronautics and astronautics at Stanford University, specialized in navigation and control systems for aerospace applications. With their extensive experience and expertise, they are well-suited to provide a comprehensive and authoritative textbook on navigation systems.

Section 8.1 describes the fundamentals of GNSS systems which will help us to have a basic understanding. It tells us how to get the distance between the satellite and the receiver.

User equipment in chapter 9 refers to the receiver. It describes the design of GNSS receivers. An GNSS receiver contains an antenna, a receiver hardware, a ranging processor, and a navigation processor. My work focuses on the tracking part of the ranging processor. This chapter points out that a common typical receiver design contains six correlators and that more than six correlators are often used in order to speed up signal capture. A complete capture flowchart is also provided to clearly describe the capture process.

Overall, as this book is a textbook, it provides a very authoritative and accurate introduction to GNSS-related technologies. It is reliable and thought-provoking.

\section{Book from E. D. Kaplan and C. Hegarty\texorpdfstring{\cite{RN177}}{}}
Elliott D. Kaplan has made significant contributions to the development and understanding of GNSS technology. Christopher J. Hegarty is also a prominent figure in the GNSS community. He has extensive experience with GPS and other satellite navigation systems. They have been associated with the MITRE Corporation and contributed to the advancement of GNSS technology and its applications.

The book delves into the fundamental principles underlying GNSS technology, including GPS, Glonass, Galileo, BeiDou, and others. It explores the concepts of satellite orbits, signal structure, receiver design, and various positioning techniques used in GNSS applications.

Chapters 3 to 7 provide a systematic introduction to the various navigation systems, including: GPS, Glonass, Galileo, BeiDou, etc. It is a great help for the background part of the project. In Chapter 8, it details the design of the receiver and introduces the antenna, RF front-end, acquisition and tracking respectively, as well as the design of the loop filter. This part was extremely helpful to my understanding. It describes the GNSS working principle and process through numerous mathematical equations.

\section{Paper from J. C. Juang, Y. H. Chen et al.\texorpdfstring{\cite{RN147}}{}}
The paper titled "\textit{Design and implementation of an adaptive code discriminator in a DSP/FPGA-based Galileo receiver}" is authored by Jyh Ching Juang, Yu Hsuan Chen, Tsai Ling Kao, and Yung Fu Tsai from National Cheng Kung University.

In their work on GNSS receiver design, the tracking performance heavily relies on the coded tracking loop and the associated discriminator. To enhance the tracking performance, the authors proposed a scheme that was implemented on a digital signal processor/field-programmable gate array board. Experiments were conducted using the GIOVE-A signal for testing, and the results demonstrated the advantages of their proposed code tracking architecture and discriminator design.

The paper presents a well-defined design methodology for an adaptive coding discriminator, which was subsequently implemented and thoroughly tested. This design approach is framed as an optimization problem, leading to the development of two discriminators. Notably, the adaptive \textit{noncoherent multi-correlator} (NMC) discriminator showcased improvements in transient response and tracking error.

Throughout the article, detailed mathematical principles of capture, tracking, and other processes are elaborated, along with a description of the correlator's design ideas. Numerous test results for correlators based on the new approach are provided, making this paper a valuable reference for my MSc project.

\section{Paper from O. Jakubov, P. Kovar et al.\texorpdfstring{\cite{RN155}}{}}
To develop advanced GNSS signal processing algorithms, such as multi-constellation, multi-frequency, and multi-antenna navigation, a flexible and re-programmable \textit{software-defined radio} (SDR) solution is essential. To achieve this goal, the researchers have introduced various receiver architectures. Their chosen approach involves constructing an RF front-end with an FPGA universal correlator, mounted on an ExpressCard directly connected to a PC. This setup allows GNSS researchers and engineers to write signal processing algorithms (e.g., tracking, acquisition, and localization) in a Linux application programming interface. The unique hardware configuration enables easy modification of the RF front-end via a PC program, providing the flexibility to increase the number of RF channels, correlators, or antennas by attaching more ExpressCards to the PC, thus enhancing computational capability.

Utilizing the SDR platform, they have successfully implemented a GNSS receiver, offering significant advantages, including low cost, high software customization, and the potential for straightforward upgrades in computational power. The article also discusses a generic FPGA-based GNSS receiver, allowing a comparison of the different approaches' advantages and disadvantages.

The outcome of their work is the prototype of the Witch Navigator receiver. They have conducted successful tests on Galileo E5, E1b, and E1c signals with the correlator and RF front-end. Moreover, they have fully addressed the communication between the FPGA and the PC, developing and testing all the corresponding controllers and drivers.

The article provides a comprehensive analysis of the design principles for receivers based on FPGAs, and the SDR platform enables validation of the FPGA platform through MATLAB simulations. This combination of approaches facilitates the development of advanced and adaptable GNSS signal processing algorithms for future navigation systems.

\subsection{Presentation Sildes from J. Parker\texorpdfstring{\cite{RN146}}{}}
This presentation was delivered at the "\textit{International Colloquium on Scientific and Fundamental Aspects of GNSS}" on 4$^{th}$ Sep. 2019. The author, Joel Parker, is a flight dynamics engineer at NASA Goddard Space Flight Center, with expertise in spacecraft mission design, astrodynamics, space navigation, software development, and space policy. He actively contributes to the flight dynamics team for the "\textit{Transiting Exoplanet Survey Satellite}" (TESS) mission, which aims to search for planets in the habitable zones of other stars and commenced in 2017.

The presentation provides a comprehensive overview of the history and scope of GNSS/GPS, as well as its potential future applications in lunar exploration and beyond.

Numerous flight examples are presented to illustrate the indispensable role of GPS in the space sector. This underscores the critical importance of GNSS technology in space missions. The author then delves into new areas of GPS application, specifically focusing on lunar exploration. The concept of a GNSS receiver designed for the Artemis Project is introduced.

The report concludes with a summary of key development directions:
\begin{itemize}
    \item Research into GNSS capabilities and enhancements for lunar applications.
    \item Collaboration with the user and supplier community, both internally and through the ICG (\textit{International Committee on GNSS}), to ensure signal performance and availability of relevant data.
    \item Utilizing proven receiver and antenna technologies to address technical challenges.
    \item Conducting flight demonstrations in the global expansion of lunar exploration activities.
    \item Working with operational programs to maximize the benefits of science and exploration.
\end{itemize}

The presentation showcases a high level of professionalism and will undoubtedly prove valuable for your MSc project, particularly in providing a thorough introduction to the background of GNSS/GPS and its ongoing development.
