\myClearDoublePage
\chapter{Methodology}
The methodology chapter is a pivotal component of this research, unveiling the systematic approach taken to address the specific goals and inquiries of the project. Within this chapter, an in-depth description of the research design, data collection methodologies, and analytical techniques deployed is meticulously presented. By offering a transparent and methodical account of the research process, this chapter safeguards the robustness and integrity of the study's outcomes.

% \section{FPGA}

\section{Tracking}
A complete GPS/GNSS work process includes acquisition, tracking and navigation as shown in figure \ref{fig:receiver}. My project is mainly focused on the tracking stage.
\begin{figure}[!h]
    \centering
    \includesvg[width=0.8\textwidth]{_IMAGES/PPT2SVG/receiver.svg}
    \caption{Diagram of the GPS/GNSS Receiver}
    \label{fig:receiver}
\end{figure}

After the acquisition phase, where the receiver identifies and locks onto the satellite signals, the tracking phase begins. During this phase, the receiver closely monitors the received signals and tracks their variations to accurately determine the user's position, velocity, and timing information. This involves maintaining a stable lock on the satellite signals despite various challenges, such as signal degradation due to atmospheric conditions, obstructions, and interference.

The following figure \ref{fig:tracking_digram} shows the basic architecture of the tracking module. The signal is received by the RF front-end and is filtered and fed to the ranging processor. At the same time, the NCO (\textit{Numerically Controlled Oscillator}) generates a GPS L1 IF carrier and a carrier for the C/A (\textit{Coarse Acquisition}) code. The GPS signal is first modulated to IF and then multiplied with the C/A code and finally integrated. Using the principle of the cross-correlation function, the C/A carrier frequency is continuously adjusted to synchronize with the C/A code in the GPS signal. And finally, complete the tracking\cite{RN151}.
\begin{figure}[!h]
    \centering
    \includesvg[width=0.8\textwidth]{_IMAGES/PPT2SVG/tracking_diagram.svg}
    \caption{Tracking Module Architecture}
    \label{fig:tracking_digram}
\end{figure}

\subsection{NCO}
In digital communications, it is often necessary to modulate and shift baseband signals (often IF) to high frequencies for transmission because the wavelengths of the high-frequency signals are better matched to the available antenna sizes. In order to modulate, we need to generate a carrier for the high-frequency signal, which is often a sine or cosine signal. Therefore, a module is required to generate the carrier at the desired frequency consistently and accurately.

Using hardware, we have three ways to generate such signals: direct form oscillator, NCO, and CORDIC algorithm. After comparing, we learnt that using the CORDIC takes up the least amount of resources in FPGAs\cite{RN181}, however, using the NCO is the simplest solution. In this project, we have designed an NCO module in FPGA to generate different frequency carriers.

The NCO is a signal generator that produces a specified frequency. It can generate square wave signals, i.e. PWM signals with a duty cycle of 50\%, as well as sinusoidal cosine signals and so on\cite{RN189}. It is often used in conjunction with a DAC (\textit{Digital-to-analog Converter}) so that an analogue signal of a specified frequency can be output. In general, the NCO consists of two parts, the \textit{Phase Accumulator} (PA) and the \textit{Look-up Table} (LUT)\cite{RN191-1}. Their architecture is shown in figure \ref{fig:nco}.

\begin{figure}[!h]
    \centering
    \includesvg[width=0.8\textwidth]{_IMAGES/PPT2SVG/NCO.svg}
    \caption{NCO Architecture}
    \label{fig:nco}
    \footnotesize Note: *FSW: \textit{Frequency Setting Word}
\end{figure}

\subsubsection{Phase Accumulator}
\label{subsec:pa_subsubsection}
The phase accumulator will complete an accumulation in each clock cycle according to the set increment value. In other words, the accumulator outputs the address of the look-up table so that the look-up table generates the correct sine-cosine signal\cite{RN191}. By setting the increment value, the look-up table is made to sample in a controlled manner.

Here, we assume that the bit width of the accumulator is 32 bits. The ratio of the incremental 32-bit value to the fixed \num{4294967296} accumulator overflow value determines how often the overflow occurs. This controls the triggering of the NCO output waveform. The equation is as follows,
\begin{equation}
    F_{out} = (increment\ value) \times \frac{F_{clock}}{2^{Accum\ width}}
    \label{equ:nco}
\end{equation}

For example, if the clock is 99.375MHz and the NCO is required to generate a 1.023MHz signal with a register bit width of 32 bits, then the increment value \num{44213852} needs to be written to FSW.

\subsubsection{Phase-to-Amplitude Converter}
By using the look-up table, we can convert the phase value into an outputable sine-cosine signal. In other words, the look-up table is actually a \textit{Phase-to-Amplitude Converter} (PAC). The easiest way to build it is to use \textit{Read-only Memory} (ROM)\cite{RN190}, create the amplitude data in advance using software such as MATLAB, and then import it into a memory file or directly into the VHDL code.

This look-up table contains all the magnitude values in a cycle. The magnitude values corresponding to the phases are rounded and stored in the table. This table can be thought of as a matrix. Assume that the bit width of each element is N bits. N is an integer power of 2. The calculation of the indexes in the table will be greatly simplified\cite{RN194}. In this case, a sine-cosine signal can be output simply by selecting the appropriate m bits in the phase representation.

In fact, we can save resources by \dquotes{cunningly} designing look-up tables\cite{RN193}. Such an operation also allows us to easily control the resolution of the NCO output. Below I will describe the method of designing a look-up table.

Firstly, according to the subsection \ref{subsec:pa_subsubsection}(Phase Accumulator), we are able to plot the phase function. Figure \ref{fig:phase_function}Below is the phase function that I am plotting using MATLAB.

The phase will be accumulated at each cycle until it overflows and starts again. In fact, this function completes the conversion from the time domain to the phase domain. Next, we will convert the phase domain to the amplitude domain, which is the PAC. The following figure also shows the PAC function generated using MATLAB.

\begin{figure}[!htbp]
    \centering
    \includesvg[width=0.8\textwidth]{_IMAGES/PPT2SVG/phase_function.svg}
    \caption{Phase Function}
    \label{fig:phase_function}
\end{figure}

\begin{figure}[!htbp]
    \centering
    \includesvg[width=0.8\textwidth]{_IMAGES/PPT2SVG/pac_function.svg}
    \caption{PAC Function}
    \label{fig:PAC_function}
\end{figure}

In figure \ref{fig:PAC_function}, one prefers to use the cosine function with high resolution, i.e. the blue line. This then consumes a lot of resources as we need to set the magnitude values for each phase and the size of the look-up table will become huge. And then such a high resolution is redundant for us. So, when we don't need such high resolution, we can \dquotes{cunningly} design the look-up table to reduce the resolution, e.g. the red lines.

Assuming, the phase overflow value is \num{1024}, i.e. 10 bits wide, and we need 8 dots per cycle, then one dot can be generated every 128 phases. Specifically, each phase interval holds a different magnitude value. Then, we can calculate that the lookup table or ROM has a depth of 8 and a width of 10 bits, occupying 80 bits.
In fact, the first interval is $0\sim127$, i.e. $00\ 0000\ 0000_{(2)}\sim00\  0111\ 1111_{(2)}$, the next interval is $00\ 1000\ 0000_{(2)}\sim00\ 1111\ 1111_{(2)}$, and so on. We can find that we only need to know the highest three bits to obtain the current amplitude value. Then, we can calculate that the depth of the ROM is 8 and the width is 3 bits, occupying 24 bits. Using this approach can greatly reduce resource usage, as shown in the table below.

\begin{table}[!htbp]
\centering
\caption{ROM Size for Different Methods}\label{tab:rom_size}
\renewcommand\arraystretch{1.5}
\begin{tabular}{ccccc}
    \toprule
    Method & ROM depth & ROM width(bits) & Size(bits) & \thead{Size percentage change\\from previous method(\%)} \\
    \midrule
    Original method & 1024 & 10 & 10240 & N/A \\
    Method 1 & 8 & 10 & 80 & -99.22 \\
    Method 2 & 8 & 3 & 24 & -70 \\
    \bottomrule
\end{tabular}
\end{table}

\subsection{Code Generator}
In GPS L1, there are two types of ranging codes, C/A code and P(\textit{Precise}) code. C/A code sends \num{1023} chips per 1ms and P code sends \num{10230} chips per 1ms, which has a higher frequency and hence is more precise. Generally, P-codes are provided to military users and are encrypted for transmission. The encrypted P-code is known as the Y-code\cite{RN195}.

Different satellites require different unique C/A codes with good correlation and balancing properties. Therefore, the use of PRN(\textit{Pseudorandom Noise}) codes is a natural fit\cite{RN197}.

Signals encoded using PRN achieve established levels of coding performance with good compatibility. As you can see from its name, the PRN code is not really a random code. It can be pre-calculated. It possesses a very small value of cross-correlation or auto-correlation. However, using it for signal processing becomes more complicated\cite{RN200}.

% 生成方法
Here are three ways to generate C/A codes\cite{RN196}, the PRN codes for GPS,
\begin{itemize}
    \item \textit{Linear Feedback Shift Register} (LFSR)
    \item Memory codes
    \item Hash functions
\end{itemize}
In fact, the most common way is to use LFSR or memory codes. 

\subsubsection{LFSR}
In simple, LFSR is a shift register. In order to generate the C/A code, we need two shift registers, called G1 generator and G2 generator. The result of exclusive-or(i.e., modulo-2 addition) between the output of G1 and the output of G2 is the C/A code. Before the exclusive-or, G2 needs to be delayed. The amount of delay is different for each satellite and this data can be queried in the ICD.

\begin{figure}[!htbp]
    \centering
    \includesvg[width=0.8\textwidth]{_IMAGES/PPT2SVG/prn2.svg}
    \caption{C/A Code Generator}
    \label{fig:ca_gen}
\end{figure}

Figure \ref{fig:ca_gen} illustrates the structural scheme of the generator. The RPN code design specification requires that the G1 shift register needs to feed the modulo-2 sums of the third and tenth levels back to the first level. Therefore, the polynomial generator for G1 is $G1=1+X^3+X^{10}$. In order for the G2 shift register to generate the specified delay, we need to feed the specified levels back to the first level after the modulo-2 addition operation according to the PRN assignments table. PRN No.2 is used here as an example, and after consulting the assignments table\cite{RN170}, we know that the phase selection for C/A is: $3\xor 7$. Therefore, we need to modulo-2 sum add the third and seventh stages as the output of G2. The polynomial of G2 is $G2=1+X^2+X^3+X^5+X^8+X^9+X^{10}$. Finally, the outputs of G1 and G2 are modulo-2 added, and the result is the C/A code, or PRN \textnumero 2 code. The clock port is driven by the 1.023MHz signal.

\subsubsection{Memory Codes}
Nowadays, in order to save the computational resources of the processor, we can store the pre-generated \num{1023}-bit PRN code in inexpensive ROM.

\subsubsection{Pros and Cons of the LFSR and Memory Codes}
LFSR:
\begin{itemize}
    \item + Known mathematical properties (balance, correlation properties) across families of codes
    \item + Simple generation - one chip at a time with small resource requirements
    \item - Limited in number
\end{itemize}
Memory codes:
\begin{itemize}
    \item + Potentially optimal
    \item - Complex selection
    \item - Requires memory on transmitter and receiver (all codes stored in non-volatile memory, possible latency issues)
\end{itemize}
In my project, I will use MATLAB to calculate the required PRN code in advance and store it in the ROM data type of VHDL

\subsection{Correlator}
The correlator is the core of the tracking module. It is used to track the received signal and generate the early, prompt, and late codes. The correlator is also called the \textit{Code Tracking Loop} (CTL). The most important thing in a correlator is to perform correlation operations, which in this case involve cross-correlation operations.

\subsubsection{Cross-Correlation}
Correlation functions are divided into two categories: auto-correlation functions and cross-correlation functions.
Correlation is a matching process that gives the similarity of two signals \cite{RN202}, defined by equation \ref{equ:auto_corr}.
\begin{equation}
    R_f(\tau)=\int_{-\infty}^{\infty}f(t)f(t+\tau)dt \quad for -\infty<\tau<\infty
    \label{equ:auto_corr}
\end{equation}

Where $x(t)$ is the signal itself. When the functions in the integrated function are the same, the $R_x(\tau)$ is then called an auto-correlation function which measures how closely a signal matches a copy of itself and the delay between these two signals.

If two functions or signals are not equal, they are said to be cross-correlation functions \cite{RN201}, defined as,
\begin{equation}
    R_{f\star g}(\tau)=\int_{-\infty}^{\infty}f(t)g(t+\tau)dt \quad for -\infty<\tau<\infty
    \label{equ:cross_corr}
\end{equation}

When the signal is discrete, equation \ref{equ:cross_corr} should be transformed into equation \ref{equ:discrete_cross_corr}. In fact, it's the scalar product of the two signals. 
\begin{equation}
    R_{f\star g}[n]=\sum_{m=-\infty}^{\infty}f(m)g(m+n) \quad for -\infty<m<\infty
    \label{equ:discrete_cross_corr}
\end{equation}

\begin{figure}[!htbp]
    \centering
    \includesvg[width=0.8\textwidth]{_IMAGES/PPT2SVG/corre_function.svg}
    \caption{Comparison of Auto-correlation and Cross-correlation}
    \label{fig:corre_function}
\end{figure}

Figure \ref{fig:corre_function} shows a visual comparison of auto-correlation and mutual correlation. Where the index of the peak of the cross-correlation is the delay value.

That is, as two signals become more similar their correlation function becomes larger. In GPS/GNSS receivers, we need to use the cross-correlation function to measure the delay between the GPS signal and the local code to synchronise.

\subsubsection{Correlator}
The correlator is a digital circuit that performs the correlation operation between the received signal and the local code. The correlation operation is an inner-production operation. The correlation operation is performed by multiplying the received signal with the local code and summing them up.

\begin{figure}[!htbp]
    \centering
    \includesvg[width=0.8\textwidth]{_IMAGES/PPT2SVG/correlator_diagram.svg}
    \caption{Architecture of the Correlator}
    \label{fig:correlator}
\end{figure}

The correlation operation usually needs 6 correlators to perform the correlation operation at the same time. In this case, the signal is processed in the I/Q phase respectively and each phase is required to multiply by the early, prompt, and late code respectively as well. Therefore, there will be 6 correlators. The massively parallel correlator arrays in receivers allow for the parallel search of thousands of code bins which will save a lot of time\cite{RN178}. The structure of the GPS correlator is shown in figure \ref{fig:correlator}.

Figure \ref{fig:code_delay} illustrates how the early, prompt and late code change as the phases of the C/A code signals are advanced with respect to the input signal. For ease of understanding, only the incoming signal is shown along with the 1-bit C/A code. In reality, the incoming PRN code is drowned in noise, and each correlator must accumulate a large amount, i.e., 1 ms of inner product, so that each envelope amplitude emerges from the noise.

\begin{figure}[!htbp]
    \centering
    \includesvg[width=0.8\textwidth]{_IMAGES/PPT2SVG/code_delay.svg}
    \caption{Code Correlation Phases}
    \label{fig:code_delay}
\end{figure}

Successful tracking is proved only when the value of P is maximum and the value of E is equal to the value of L.

\subsection{Doppler Effect}
A typical scenario is one in which the receiver antenna is fixed to the earth and constantly receives signals from satellites in orbit. Using the antenna as a reference system, the satellite is constantly moving at high speed. The frequency of the signal from the satellite is fixed, while the frequency received on the ground is constantly changing, a phenomenon called the Doppler effect, which produces a Doppler shift.

The Doppler shift is,
\begin{equation}
    \nu =\frac{v}{\lambda}
\end{equation}

Where $v$ is the relative velocity between the transmitter and receiver, which is negative if its distance becomes small and positive if the distance becomes large. $\lambda$ is the wavelength of the signal \cite{RN203}. In this example, however, the transmitter and receiver are moving in the same straight line. When this requirement is not met, we need to revise the formula.

Assuming that the receiver is moving at the speed of $v$, the angle between the moving direction and the wave propagation direction is $\gamma$ as shown in figure \ref{fig:doppler}.

\begin{figure}[!htbp]
    \centering
    \includesvg[width=0.7\textwidth]{_IMAGES/PPT2SVG/doppler.svg}
    \caption{Diagram of the Doppler Effect in Two-dimensional}
    \label{fig:doppler}
\end{figure}

In this case, the speed of movement has changed to the projection of the speed of movement, which is $v\cos(\gamma)$. The revised Doppler shift is then:
\begin{equation}
    \nu =\frac{v}{\lambda}\cos(\gamma)
\end{equation}

\section{Verification}
\subsection{Software Verification}
To verify the correctness of the results, this project will use open-source code \textit{SoftGNSS} for Matlab simulation. Also, the results of the simulation will help me to debug the code better.

\subsection{Hardware Verification}
Verification is an important step in FPGA development that saves time by quickly identifying faults in the code. It also prevents faults from being carried over to the next stage of the chip such as ASICs, which saves a lot of cost \cite{RN204}. There are three main verification methods for developing FPGAs: behavioural simulation, post-synthesis functional simulation and post-implementation functional simulation\cite{RN205}. Due to the time constraints of the project, only behavioural simulation was used for this project.
\begin{itemize}
    \item Behavioural Simulation:\\RTL-level simulation allows us to simulate and verify your design before any transformations are performed by synthesis or implementation tools. We can verify your design as a module, block, or system.\\
    Typically, RTL simulation is performed to verify code syntax and confirm that the code functions as expected. In this step, the design is described primarily in RTL, so no timing information is required.
    \item Post-Synthesis Functional Simulation:\\It allows simulation of synthesized netlists to verify that the synthesized design meets functional requirements and behaves as expected.
    \item Post-Implementation Functional Simulation:\\We can perform either functional simulation or timing simulation after implementation. Timing simulation is the closest simulation to actually downloading the design into the device. It allows you to ensure that the implemented design meets the functional and timing requirements and that the behaviour within the device matches the desired behaviour.
\end{itemize}

