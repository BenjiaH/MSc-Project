\chapter{Methodology}
The methodology chapter is a pivotal component of this research, unveiling the systematic approach taken to address the specific goals and inquiries of the project. Within this chapter, an in-depth description of the research design, data collection methodologies, and analytical techniques deployed is meticulously presented. By offering a transparent and methodical account of the research process, this chapter safeguards the robustness and integrity of the study's outcomes.

% \section{GPS L1 Signal Plan}
% As mentioned in the previous chapter, the GPS L1 signal is our target signal. One of the reasons is that the GPS was the first positioning  system to come online\cite{RN184}. The other reason is that the GPS L1 signal is the most widely used signal in the world.

% Based on the GPS L1 ICD\cite{RN170}, we can summarize the signal plan of GPS L1 in table \ref{tab:spec_gpsl1}.

% \begin{table}[!htbp]
% \centering
% \caption{Specifications of GPS L1}\label{tab:spec_gpsl1}
% \renewcommand\arraystretch{1.3}
% \begin{tabular}{cccc}
%     \toprule
%     Specifications & \multicolumn{3}{c}{GPS} \\
%     \midrule
%     Service Name & C/A & \multicolumn{2}{c}{LIC} \\
%     Centre Frequency & 1575.42MHz & \multicolumn{2}{c}{1575.42 MHz} \\
%     Frequency Band & L1 & \multicolumn{2}{c}{L1} \\
%     Access Technique & CDMA & \multicolumn{2}{c}{CDMA} \\
%     Signal Component & Data & Data & Pilot \\
%     Modulation & BPSK & \multicolumn{2}{c}{TMBOC(6,1,1/11)} \\
%     % Sub-carrierfrequency {[}MHz{]} & N/A & 1.023 & 1.023\&6.138 \\
%     Code Frequency & 1.023 MHz & \multicolumn{2}{c}{1.023 MHz} \\
%     Primary PRN Code Length & 1023 & \multicolumn{2}{c}{10230} \\
%     Code Family & Gold Codes & \multicolumn{2}{c}{Weil Codes} \\
%     Data Rate & 50 bps/50 sps & 50 bps/100 sps & N/A \\
%     \bottomrule
% \end{tabular}
% \end{table}




\section{Tracking}
A complete GPS/GNSS work process includes acquisition, tracking and navigation. My project is mainly focused on the tracking stage.

After the acquisition phase, where the receiver identifies and locks onto the satellite signals, the tracking phase begins. During this phase, the receiver closely monitors the received signals and tracks their variations to accurately determine the user's position, velocity, and timing information. This involves maintaining a stable lock on the satellite signals despite various challenges, such as signal degradation due to atmospheric conditions, obstructions, and interference.

The following figure \ref{fig:tracking_digram} shows the basic architecture of the tracking module. The signal is received by the RF front-end and is filtered and fed to the ranging processor. At the same time, the NCO (\textit{Numerically Controlled Oscillator}) generates a GPS L1 IF carrier and a carrier for the C/A (\textit{Coarse Acquisition}) code. The GPS signal is first modulated to IF and then multiplied with the C/A code and finally integrated. Using the principle of the cross-correlation function, the C/A carrier frequency is continuously adjusted to synchronize with the C/A code in the GPS signal. And finally, complete the tracking\cite{RN151}.
\begin{figure}[!h]
    \centering
    \includesvg[width=0.7\textwidth]{_IMAGES/PPT2SVG/tracking_diagram.svg}
    \caption{Tracking Module Architecture}
    \label{fig:tracking_digram}
    % \footnotesize Note: *NCO: \textit{Numerically Controlled Oscillator}
\end{figure}

\subsection{NCO}
In digital communications, it is often necessary to modulate and shift baseband signals (often IF) to high frequencies for transmission because the wavelengths of the high frequency signals are better matched to the available antenna sizes. In order to modulate, we need to generate a carrier for the high frequency signal, which is often a sine or cosine signal. Therefore, a module is needed to generate the carrier at the desired frequency consistently and accurately.

For hardware, we have three ways to generate such signals: direct form oscillator, NCO, and CORDIC algorithm. After comparing, we learnt that using the CORDIC takes up the least amount of resources in FPGAs\cite{RN181}, however, using the NCO is the simplest solution. In this project we have designed NCO module in FPGA to generate different frequency carriers.
% \cite{RN189}  \cite{RN190} \cite{RN191} 

\subsection{Code Generator}



\subsection{Correlator}
\subsubsection{Cross-Correlation}

\subsection{Doppler}

\section{Verification}
\subsection{Behavioural Simulation}
\subsection{Post-Synthesis Functional Simulation}
\subsection{Post-Implementation Functional Simulation}
