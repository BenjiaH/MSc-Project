\myClearDoublePage
\chapter{Implementation}

The previous section focused on the theoretical approach to designing a tracking module for a receiver, and this chapter will explain the specific practical process and the problems encountered during the process. This project mainly implements the tracking module on a GPS receiver, please note that this is only a separate module and does not realise the full receiver functionality. It also has RF front-end and acquisition modules in the front module, and ranging modules as well as navigation modules in the back module. The project in fact has a hypothetical target device, but based on the current conditions, the main use of VHDL for code writing, as well as behavioural simulation to verify the results, will not be on-board operations.

\section{System Architecture}
\subsection{RF Front-end}

The RF front-end is a device that collects any signals you expect. In this project, we will use \textit{NT1065\_FMC2} as our front-end. This device is designed to receive GPS, GLONASS, Galileo, BeiDou, IRNSS, QZSS and L1, L2, L3, L5, E1, E5a, E5b, E6, B1, B2, B3 bands. It has an FMC(\textit{FPGA Mezzanine Card}) interface, which allows it to have a faster transfer rate to the Xilinx board \cite{RN206}. Here is its key specification table.

\begin{table}[!htbp]
\centering
\caption{Key Specification of \textit{NT1065\_FMC2}}\label{tab:nt1065}
\renewcommand\arraystretch{1.5}
\begin{tabular}{cc}
    \toprule
    Content & Specification \\
    \midrule
    Chip & NT1065 \\
    Number of channels & 4 \\
    \multirow{2}{*}{Reference frequency sources} & 10 MHz TCXO \\
     & 24.84 MHz TCXO \\
    Bit width of ADC & 2 bits \\
    \bottomrule
\end{tabular}
\end{table}


\subsection{FPGA Board}

\section{Develop Environment}
\subsection{Vivado}
\subsection{Modelsim}
\subsection{MATLAB}

\section{}



\subsection{Acquisition}

Vivado

